%%%%%%%%%%%%%%%%%%%%%%%%%%%%%%%%%%%%%%%%%
% Tufte-Style Book (Minimal Template)
% LaTeX Template
% Version 1.0 (5/1/13)
%
% This template has been downloaded from:
% http://www.LaTeXTemplates.com
%
% License:
% CC BY-NC-SA 3.0 (http://creativecommons.org/licenses/by-nc-sa/3.0/)
%
% IMPORTANT NOTE:
% In addition to running BibTeX to compile the reference list from the .bib
% file, you will need to run MakeIndex to compile the index at the end of the
% document.
%
%%%%%%%%%%%%%%%%%%%%%%%%%%%%%%%%%%%%%%%%%

%----------------------------------------------------------------------------------------
%	PACKAGES AND OTHER DOCUMENT CONFIGURATIONS
%----------------------------------------------------------------------------------------

%\documentclass{tufte-book} % Use the tufte-book class which in turn uses the tufte-common class
\documentclass[ebook,11pt,openany]{book}
\parskip 10pt plus 10pt
\parindent 0pt
\topmargin -20pt
\headheight 5mm
\headsep 5mm
\oddsidemargin 0mm
\evensidemargin 0mm
\textwidth 170mm
\textheight 220mm

\usepackage{color,listings,seqsplit}
\definecolor{codegreen}{rgb}{0,0.6,0}
\definecolor{codegray}{rgb}{0.5,0.5,0.5}
\definecolor{codepurple}{rgb}{0.58,0,0.82}
\definecolor{backcolour}{rgb}{0.95,0.95,0.92}
\definecolor{Darkgreen}{rgb}{0,0.4,0}
\definecolor{NavyBlue}{rgb}{0.125,0.165,0.267}
% Definig a custom style:
\lstdefinestyle{mystyle}{
    language=C++,
    backgroundcolor=\color{backcolour},
    commentstyle=\color{Darkgreen},
    identifierstyle=\color{magenta},
    keywordstyle=\color{NavyBlue},
    numberstyle=\tiny\color{codegray},
    stringstyle=\color{codepurple},
    basicstyle=\ttfamily\footnotesize\bfseries,
    breakatwhitespace=false,
    breaklines=true,
    captionpos=b,
    keepspaces=true,
    numbers=left,
    numbersep=5pt,
    showspaces=false,
    showstringspaces=false,
    showtabs=false,
    tabsize=2
}
% -- Setting up the custom style:
\lstset{style=mystyle}

\usepackage{tocloft}
\usepackage{multirow}
\addtocontents{toc}{\setlength{\cftchapnumwidth}{2em}}
\addtocontents{toc}{\setlength{\cftsecnumwidth}{4em}}
\addtocontents{toc}{\setlength{\cftsubsecnumwidth}{4em}}

\usepackage{microtype} % Improves character and word spacing

\usepackage{booktabs} % Better horizontal rules in tables

\usepackage{graphicx} % Needed to insert images into the document
\setkeys{Gin}{width=\linewidth,totalheight=\textheight,keepaspectratio} % Improves figure scaling

\usepackage{fancyvrb} % Allows customization of verbatim environments
\fvset{fontsize=\normalsize} % The font size of all verbatim text can be changed here

\newcommand{\hangp}[1]{\makebox[0pt][r]{(}#1\makebox[0pt][l]{)}} % New command to create parentheses around text in tables which take up no horizontal space - this improves column spacing
\newcommand{\hangstar}{\makebox[0pt][l]{*}} % New command to create asterisks in tables which take up no horizontal space - this improves column spacing

\usepackage{xspace} % Used for printing a trailing space better than using a tilde (~) using the \xspace command

\newcommand{\monthyear}{\ifcase\month\or January\or February\or March\or April\or May\or June\or July\or August\or September\or October\or November\or December\fi\space\number\year} % A command to print the current month and year

\newcommand{\blankpage}{\newpage\hbox{}\thispagestyle{empty}\newpage} % Command to insert a blank page

\usepackage{makeidx} % Used to generate the index
\makeindex % Generate the index which is printed at the end of the document

\renewcommand{\thefootnote}{\arabic{footnote}}
\usepackage{fancyhdr,nestenum,supertabular,calc}
\pagestyle{fancy}

\usepackage{hyperref}
\hypersetup{colorlinks} % Comment this line if you don't wish to have colored links
